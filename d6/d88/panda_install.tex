
\begin{DoxyVerbInclude}

Installation Instructions
*************************

In order to install the PandA framework, you need to perform the following steps:

1) Give write permission to your user on /opt:
   $chown -R <username.usergroup> /opt

2) Install required packages (for Ubuntu/Debian/CentOS/Scientific Linux/ArchLinux/Fedora systems, see below)

3) Prepare the configuration script:
   $make -f Makefile.init

4) Create the directory for the building:
   $mkdir obj
   $cd obj

5) Configure PandA:
   $../configure [--enable-<ext-feature>...] --prefix=/opt/panda

Currently accepted parameters are:

  --enable-flopoco: add floating point support by leveraging FloPoCo Library 
                   http://flopoco.gforge.inria.fr/
  --enable-xilinx: enable Xilinx RTL synthesis and simulation tools execution 
                   within the framework
  --enable-altera: enable Altera RTL synthesis tool execution within the framework
  --enable-lattice: enable Lattice RTL synthesis tool execution within the framework
  --enable-nanoxplore: enable BRAVE FPGA NanoXplore RTL synthesis tool execution within the framework
  --enable-modelsim: enable Modelsim-based simulation directly within the framework
  --with-mentor-license=<license-file/license-server>
  --with-nanoxplore-license=<license-file/license-server>
  --enable-icarus: enable Icarus-based simulation
                   http://iverilog.icarus.com/
  --enable-verilator: enable Verilator-based simulation
                   http://www.veripool.org/wiki/verilator
  --enable-debug: add debugging symbols to the binaries 
  --enable-opt: Compile the framework with GCC optimizations enabled
  --enable-release: Remove further tests not useful in a production environment

Note that the scripts for synthesis and simulation are generated even if the 
corresponding tools are not configured as described above. In this case, the 
synthesis/simulation has to be performed using the standard shell command line.
Since FloPoCo generates a VHDL-based description, co-simulation requires a
mixed-language simulator. Both Modelsim from Mentor and XSIM/ISIM from Xilinx
have such support.

The suggested configuration is
../configure --prefix=/opt/panda --enable-flopoco --enable-icarus --enable-verilator --enable-xilinx --enable-modelsim --enable-altera --enable-nanoxplore --enable-opt --enable-lattice --enable-release --with-mentor-license=<license-string> --with-nanoxplore-license=<license-string>

Options --enable-xilinx, --enable-modelsim, --enable-altera, --enable-lattice, --enable-nanoxplore, --with-mentor-license=<license-string>, and --with-nanoxplore-license=<license-string> require that third-party tools have been installed before PandA is configured. Details on where these third-party tools are searched are provided at the end of this file.

7) Compile the tool:
  $make

8) Install the tool:
  $make install

At the end, if the installation has been successful, you can run the tool:
  $/opt/panda/bin/bambu

which prints its help message.

Additional documentation about the configuration, the execution of the test 
examples, and the creation of the basic documentation of the framework can be
found at: http://panda.dei.polimi.it/?page_id=88

Testcases
*********

In the directory examples, you can find several different testcases, with the 
bash scripts for their execution.

These scripts automatically create subdirectories containing all the results 
of the execution. You can start from them to build your own hardware 
accelerators with bambu.



Documentation
*************
After configuring the framework in the directory for the building, you can
create a preliminary documentation of the framework by executing the following 
command:
  $cd obj
  $make documentation

Then, you can start browsing the documentation from the file 
  obj/doc/PANDA_html/index.html


Ubuntu packages
***************

The following packages have to be installed under Ubuntu to compile PandA

Ubuntu 18.04 64bit:
- autoconf
- autoconf-archive
- automake
- libtool
- g++
- gcc-4.8
- g++-4.8
- gcc-5
- g++-5
- gcc-6
- g++-6
- gcc-7
- g++-7
- gcc-8
- g++-8
- gcc-4.8-plugin-dev
- gcc-5-plugin-dev
- gcc-6-plugin-dev
- gcc-7-plugin-dev
- gcc-8-plugin-dev
- gcc-4.8-multilib
- gcc-5-multilib
- gcc-6-multilib
- gcc-7-multilib
- gcc-7-multilib
- g++-4.8-multilib
- g++-5-multilib
- g++-6-multilib
- g++-7-multilib
- g++-8-multilib
- gfortran-4.8
- gfortran-4.8-multilib 
- gfortran-5
- gfortran-5-multilib
- gfortran-6 
- gfortran-6-multilib
- gfortran-7
- gfortran-7-multilib
- gfortran-8
- gfortran-8-multilib
- clang-4.0
- libclang-4.0-dev
- clang-5.0
- libclang-5.0-dev
- clang-6.0
- libclang-6.0-dev
- clang-7
- libclang-7-dev
- libbdd-dev
- libboost-all-dev
- libmpc-dev
- libmpfr-dev
- libxml2-dev
- liblzma-dev
- libmpfi-dev
- zlib1g-dev
- libicu-dev
- bison
- doxygen
- flex
- graphviz
- iverilog
- verilator
- pkg-config
- libsuitesparse-dev
- libglpk-dev
- make


In a single line:
sudo apt-get install autoconf autoconf-archive automake libtool g++ gcc-4.8 g++-4.8 gcc-5 g++-5 gcc-6 g++-6 gcc-7 g++-7 gcc-8 g++-8 gcc-4.8-plugin-dev gcc-5-plugin-dev gcc-6-plugin-dev gcc-7-plugin-dev  gcc-8-plugin-dev gcc-4.8-multilib gcc-5-multilib gcc-6-multilib gcc-7-multilib gcc-8-multilib g++-4.8-multilib g++-5-multilib g++-6-multilib g++-7-multilib g++-8-multilib gfortran-4.8 gfortran-4.8-multilib gfortran-5 gfortran-5-multilib gfortran-6 gfortran-6-multilib gfortran-7 gfortran-7-multilib gfortran-8 gfortran-8-multilib clang-4.0 libclang-4.0-dev libclang-6.0-dev clang-6.0 libclang-6.0-dev clang-7 libclang-7-dev libbdd-dev libboost-all-dev libmpc-dev libmpfr-dev libxml2-dev liblzma-dev libmpfi-dev zlib1g-dev libicu-dev bison doxygen flex graphviz iverilog verilator make libsuitesparse-dev libglpk-dev


Ubuntu 17.04 64bit and Ubuntu 17.04 32bit:
- autoconf
- autoconf-archive
- automake
- libtool
- g++
- gcc-4.8
- g++-4.8
- gcc-4.9
- g++-4.9
- gcc-4.8-plugin-dev
- gcc-4.9-plugin-dev
- gcc-5-plugin-dev
- gcc-6-plugin-dev
- gcc-4.8-multilib
- gcc-4.9-multilib
- gcc-5-multilib
- gcc-6-multilib
- g++-4.8-multilib
- g++-4.9-multilib
- g++-5-multilib
- g++-6-multilib
- gfortran-4.8
- gfortran-4.8-multilib 
- gfortran-4.9
- gfortran-4.9-multilib
- gfortran-5 
- gfortran-5-multilib
- gfortran-6 
- gfortran-6-multilib
- clang-4.0
- libclang-4.0-dev
- libbdd-dev
- libboost-all-dev
- libmpc-dev
- libmpfr-dev
- libxml2-dev
- liblzma-dev
- libmpfi-dev
- zlib1g-dev
- libicu-dev
- bison
- doxygen
- flex
- graphviz
- iverilog
- verilator
- pkg-config
- libsuitesparse-dev
- libglpk-dev
- make


In a single line:
sudo apt install autoconf autoconf-archive automake libtool g++ gcc-4.8 g++-4.8 gcc-4.9 g++-4.9 gcc-5 g++-5 gcc-6 g++-6 gcc-4.8-plugin-dev gcc-4.9-plugin-dev gcc-5-plugin-dev gcc-6-plugin-dev gcc-4.8-multilib gcc-4.9-multilib gcc-5-multilib gcc-6-multilib g++-4.8-multilib g++-4.9-multilib g++-5-multilib g++-6-multilib gfortran-4.8 gfortran-4.8-multilib gfortran-4.9 gfortran-4.9-multilib gfortran-5 gfortran-5-multilib gfortran-6 gfortran-6-multilib clang-4.0 libclang-4.0-dev libbdd-dev libboost-all-dev libmpc-dev libmpfr-dev libxml2-dev liblzma-dev libmpfi-dev zlib1g-dev libicu-dev bison doxygen flex graphviz iverilog verilator make libsuitesparse-dev libglpk-dev

Ubuntu 16.10 64bit:
- autoconf
- autoconf-archive
- automake
- libtool
- g++
- gcc-4.8
- g++-4.8
- gcc-4.9
- g++-4.9
- gcc-4.8-plugin-dev
- gcc-4.9-plugin-dev
- gcc-5-plugin-dev
- gcc-6-plugin-dev
- gcc-4.8-multilib
- gcc-4.9-multilib
- gcc-5-multilib
- gcc-6-multilib
- g++-4.8-multilib
- g++-4.9-multilib
- g++-5-multilib
- g++-6-multilib
- gfortran-4.8
- gfortran-4.8-multilib 
- gfortran-4.9
- gfortran-4.9-multilib
- gfortran-5 
- gfortran-5-multilib
- gfortran-6 
- gfortran-6-multilib
- clang-4.0
- libclang-4.0-dev
- libbdd-dev
- libboost-all-dev
- libmpc-dev
- libmpfr-dev
- libxml2-dev
- liblzma-dev
- libmpfi-dev
- zlib1g-dev
- libicu-dev
- bison
- doxygen
- flex
- graphviz
- iverilog
- verilator
- pkg-config
- libsuitesparse-dev
- libglpk-dev
- make


In a single line:
sudo apt install autoconf autoconf-archive automake libtool g++ gcc-4.8 g++-4.8 gcc-4.9 g++-4.9 gcc-5 g++-5 gcc-6 g++-6 gcc-4.8-plugin-dev gcc-4.9-plugin-dev gcc-5-plugin-dev gcc-6-plugin-dev gcc-4.8-multilib gcc-4.9-multilib gcc-5-multilib gcc-6-multilib g++-4.8-multilib g++-4.9-multilib g++-5-multilib g++-6-multilib gfortran-4.8 gfortran-4.8-multilib gfortran-4.9 gfortran-4.9-multilib gfortran-5 gfortran-5-multilib gfortran-6 gfortran-6-multilib clang-4.0 libclang-4.0-dev libbdd-dev libboost-all-dev libmpc-dev libmpfr-dev libxml2-dev liblzma-dev libmpfi-dev zlib1g-dev libicu-dev bison doxygen flex graphviz iverilog verilator make libsuitesparse-dev libglpk-dev

Ubuntu 16.04 64bit:
- autoconf
- autoconf-archive
- automake
- libtool
- g++
- gcc-4.8
- g++-4.8
- gcc-4.9
- g++-4.9
- g++-5
- gcc-4.8-plugin-dev
- gcc-4.9-plugin-dev
- gcc-5-plugin-dev
- gcc-4.8-multilib
- gcc-4.9-multilib
- gcc-5-multilib
- g++-4.8-multilib
- g++-4.9-multilib
- g++-5-multilib
- gfortran-4.8
- gfortran-4.8-multilib 
- gfortran-4.9
- gfortran-4.9-multilib
- gfortran-5 
- gfortran-5-multilib
- clang-4.0
- libclang-4.0-dev
- libbdd-dev
- libboost-all-dev
- libmpc-dev
- libmpfr-dev
- libxml2-dev
- liblzma-dev
- libmpfi-dev
- zlib1g-dev
- libicu-dev
- bison
- doxygen
- flex
- graphviz
- iverilog
- verilator
- pkg-config
- libsuitesparse-dev
- libglpk-dev
- make


In a single line:
sudo apt-get install autoconf autoconf-archive automake libtool g++ gcc-4.8 g++-4.8 gcc-4.9 g++-4.9 gcc-5 g++-5 gcc-4.8-plugin-dev gcc-4.9-plugin-dev gcc-5-plugin-dev  gcc-4.8-multilib gcc-4.9-multilib gcc-5-multilib g++-4.8-multilib g++-4.9-multilib g++-5-multilib gfortran-4.8 gfortran-4.8-multilib gfortran-4.9 gfortran-4.9-multilib gfortran-5 gfortran-5-multilib clang-4.0 libclang-4.0-dev libbdd-dev libboost-all-dev libmpc-dev libmpfr-dev libxml2-dev liblzma-dev libmpfi-dev zlib1g-dev libicu-dev bison doxygen flex graphviz iverilog verilator make libsuitesparse-dev libglpk-dev


Ubuntu 14.04 LTS 64bit and Ubuntu 14.04 LTS 32bit:
- autoconf
- autoconf-archive
- automake
- libtool
- g++
- gcc-4.6
- g++-4.6
- gcc-4.7
- g++-4.7
- gcc-4.8-plugin-dev
- gcc-4.7-plugin-dev
- gcc-4.6-plugin-dev
- gcc-4.6-multilib
- gcc-4.7-multilib
- gcc-4.8-multilib
- g++-4.8-multilib
- gfortran-4.6
- gfortran-4.6-multilib
- gfortran-4.7
- gfortran-4.7-multilib
- gfortran-4.8
- gfortran-4.8-multilib
- libboost-all-dev
- libmpc-dev
- libmpfr-dev
- libxml2-dev
- liblzma-dev
- libmpfi-dev
- zlib1g-dev
- libicu-dev
- bison
- doxygen
- flex
- graphviz
- iverilog
- verilator
- pkg-config
- libsuitesparse-dev
- libglpk-dev
- make


In a single line:
sudo apt-get install autoconf autoconf-archive automake libtool g++ gcc-4.6 g++-4.6 gcc-4.6-plugin-dev gcc-4.6-plugin-dev gcc-4.7 g++-4.7 gcc-4.8-plugin-dev gcc-4.7-plugin-dev gcc-4.6-multilib gcc-4.7-multilib gcc-4.8-multilib g++-4.8-multilib gfortran-4.6 gfortran-4.6-multilib gfortran-4.7 gfortran-4.7-multilib gfortran-4.8 gfortran-4.8-multilib libboost-all-dev libmpc-dev libmpfr-dev libxml2-dev liblzma-dev libmpfi-dev zlib1g-dev libicu-dev bison doxygen flex graphviz iverilog verilator libsuitesparse-dev libglpk-dev make

Debian packages
***************

The following packages have to be installed under Debian to compile PandA.

Debian unstable (sid):
- autoconf
- autoconf-archive
- libtool
- make
- g++
- gcc-5
- g++-5
- gcc-6
- g++-6
- gcc-7
- g++-7
- gcc-8
- g++-8
- gcc-5-plugin-dev
- gcc-6-plugin-dev
- gcc-7-plugin-dev
- gcc-8-plugin-dev
- gcc-5-multilib
- gcc-6-multilib
- gcc-7-multilib
- gcc-8-multilib
- gfortran
- gfortran-multilib 
- clang-4.0
- libclang-4.0-dev
- clang-5.0
- libclang-5.0-dev
- clang-6.0
- libclang-6.0-dev
- libbdd-dev
- libboost-all-dev
- libmpc-dev
- libmpfr-dev
- libxml2-dev
- liblzma-dev
- libmpfi-dev
- zlib1g-dev
- libicu-dev
- bison
- doxygen
- flex
- graphviz
- iverilog
- verilator
- pkg-config
- libsuitesparse-dev
- libglpk-dev
- make

In a single line (as root user):
apt install autoconf autoconf-archive libtool make g++ gcc-5 g++-5 gcc-6 g++-6 gcc-7 g++-7 gcc-5-plugin-dev gcc-6-plugin-dev gcc-7-plugin-dev gcc-8-plugin-dev gcc-5-multilib gcc-6-multilib gcc-7-multilib gcc-8-multilib gfortran gfortran-multilib clang-4.0 libclang-4.0-dev clang-5.0 libclang-5.0-dev clang-6.0 libclang-6.0-dev libbdd-dev libboost-all-dev libmpc-dev libmpfr-dev libxml2-dev liblzma-dev libmpfi-dev zlib1g-dev libicu-dev bison doxygen flex graphviz iverilog verilator pkg-config

Debian 9 (Stretch):
- autoconf
- autoconf-archive
- libtool
- make
- gcc
- g++
- gcc-multilib
- gcc-6-plugin-dev
- gfortran
- gfortran-multilib 
- clang-4.0
- libclang-4.0-dev
- libbdd-dev
- libcloog-ppl1
- libboost-all-dev
- libmpc-dev
- libmpfr-dev
- libxml2-dev
- liblzma-dev
- libmpfi-dev
- zlib1g-dev
- libicu-dev
- bison
- doxygen
- flex
- graphviz
- iverilog
- verilator
- pkg-config
- libsuitesparse-dev
- libglpk-dev

In a single line (as root user):
apt install autoconf autoconf-archive libtool make gcc g++ gcc-multilib gcc-6-plugin-dev gfortran gfortran-multilib libcloog-ppl1 clang-4.0 libclang-4.0-dev libbdd-dev libboost-all-dev libmpc-dev libmpfr-dev libxml2-dev liblzma-dev libmpfi-dev zlib1g-dev libicu-dev bison doxygen flex graphviz iverilog verilator pkg-config libsuitesparse-dev libglpk-dev

CentOS 6/Scientific Linux distributions
*************
These two distributions are very similar and present the very same problems. They use an old gcc compiler, not compatible with the PandA framework and they do not have a compatible version of boost libraries. So these are few notes for those who are still willing to install the PandA framework on such systems.
First, some dependencies need to be satisfied:
   $ sudo yum install python-devel zlib-devel bzip2-devel libicu-devel glibc-devel.i686 libgcc.i686 wget tar bzip2 gcc glibc-devel gcc-c++ suiteparse-devel libtool-ltdl-devel glpk-devel

autoconf-archive package is also required, but it is not present in CentOS repositories:

   $ wget ftp://ftp.pbone.net/mirror/ftp5.gwdg.de/pub/opensuse/repositories/home:/pelliott11:/autoconf-archive/CentOS_CentOS-6/noarch/autoconf-archive-2012.04.07-7.3.noarch.rpm
   $sudo rpm -Uvh autoconf-archive-2012.04.07-7.3.noarch.rpm

Then a recent version of boost has to be installed by hand.

   $ wget http://sourceforge.net/projects/boost/files/boost/1.48.0/boost_1_48_0.tar.bz2/download
   $ cp download boost_1_48_0.tar.bz2
   $ tar xf boost_1_48_0.tar.bz2 -C /tmp
   $ cd /tmp/boost_1_48_0/
   $ sudo mkdir -p /opt/boost/1.48.0
   $ sudo ./bootstrap.sh
   $ sudo ./bjam --prefix=/opt/boost/1.48.0/ variant=release install

Once you have boost installed, you need a recent gcc to compile the PandA sources. This can be done with devtools

On CentOS you can retrieve the devtools repos with the following line:

   $ sudo wget http://people.centos.org/tru/devtools-1.1/devtools-1.1.repo -O /etc/yum.repos.d/devtools-1.1.repo

On Scientific Linux you can follow the instructions at this link http://linux.web.cern.ch/linux/devtoolset/ to retrieve the repos info.

Then for gcc-4.7.2 run the following line:

   $ sudo yum install devtoolset-1.1 devtoolset-1.1-runtime devtoolset-1.1-binutils devtoolset-1.1-binutils-devel devtoolset-1.1-build devtoolset-1.1-dwz devtoolset-1.1-elfutils devtoolset-1.1-elfutils-devel devtoolset-1.1-elfutils-libelf devtoolset-1.1-elfutils-libelf-devel devtoolset-1.1-elfutils-libs devtoolset-1.1-gcc devtoolset-1.1-gcc-c++ devtoolset-1.1-gcc-gfortran devtoolset-1.1-gcc-plugin-devel devtoolset-1.1-gdb devtoolset-1.1-libitm-devel devtoolset-1.1-libquadmath-devel devtoolset-1.1-libstdc++-devel devtoolset-1.1-valgrind

After this you can enable gcc-4-7 by typing:

   $ scl enable devtoolset-1.1 bash

Finally the following packages have to be installed:
   $ sudo yum install automake bison flex libtool libxml2-devel mpfr-devel

This enviroment allows the compilation of the PandA framework with a minimal configuration without flopoco library and without Icarus verilog simulator.
The configure we tested for centos is the following one:

   $ CXX=/opt/centos/devtoolset-1.1/root/usr/bin/g++ CC=/opt/centos/devtoolset-1.1/root/usr/bin/gcc ../configure --prefix=/opt/panda --enable-opt --enable-release --with-boost=/opt/boost/1.48.0/ --with-gcc47=/opt/centos/devtoolset-1.1/root/usr/bin/gcc

where --with-boost specifies the manually installed boost library and --with-gcc47 specifies the gcc executable provided by devtools-1.1.
FloPoCo is currently not supported by bambu under CentOS 6.

Note that, before the configure, you need to run the make -f Makefile.init command as described in the standard PandA *Installation Instructions*.

CentOS 7
***************

The following packages have to be installed under Centos 7 to compile PandA:
- autoconf
- autoconf-archive
- automake
- bison
- boost-devel
- flex
- gcc-c++
- gcc-plugin-devel
- glibc-devel.i686
- glpk-devel
- libgcc.i686
- libicu-devel
- libtool
- libtool-ltdl-devel
- libxml2-devel
- make
- mpfr-devel
- suitesparse-static
- verilator
- xz-devel
- zlib-devel

Extra packages for Enterprise Linux (EPEL) is required:
yum install epel-release

In a single line (as root):
yum install autoconf autoconf-archive automake bison boost-devel flex gcc-c++ gcc-plugin-devel glibc-devel.i686 glpk-devel libgcc.i686 libicu-devel libtool libtool-ltdl-devel libxml2-devel make mpfr-devel suitesparse-static verilator xz-devel zlib-devel

FloPoCo is currently not supported by bambu under CentOS 7.

Fedora
***************

The following packages have to be installed under Fedora 23 (64 bit) to compile PandA:
sudo dnf install autoconf autoconf-archive libtool boost-devel mpfr-devel xml2 lzma-devel mpfi-devel zlib-devel bison flex doxygen graphviz iverilog verilator gcc gcc-c++ gcc-plugin-devel.x86_64 gcc-plugin-devel.i686 libxml2-devel glibc glibc-devel glibc.i686 glibc-devel.i686 libgcc.i686 make suitesparse-static libtool-ltdl-devel glpk-devel gawk

The following packages have to be installed under Fedora 24 (64 bit) to compile PandA:
sudo dnf install autoconf autoconf-archive libtool boost-devel mpfr-devel xml2 lzma-devel mpfi-devel zlib-devel bison flex doxygen graphviz iverilog verilator gcc gcc-c++ gcc-plugin-devel.x86_64 gcc-plugin-devel.i686 libxml2-devel glibc glibc-devel glibc.i686 glibc-devel.i686 libgcc.i686 make suitesparse-static libtool-ltdl-devel glpk-devel gawk which

The following packages have to be installed under Fedora 25 (64 bit) to compile PandA:
sudo dnf install autoconf autoconf-archive libtool boost-devel mpfr-devel xml2 lzma-devel mpfi-devel zlib-devel bison flex doxygen graphviz iverilog verilator gcc gcc-c++ gcc-plugin-devel.x86_64 gcc-plugin-devel.i686 libxml2-devel glibc glibc-devel glibc.i686 glibc-devel.i686 libgcc.i686 make suitesparse-static libtool-ltdl-devel glpk-devel gawk

The following packages have to be installed under Fedora 29 (64 bit) to compile PandA:
sudo dnf install autoconf autoconf-archive libtool boost-devel mpfr-devel xml2 lzma-devel mpfi-devel zlib-devel bison flex doxygen graphviz iverilog verilator gcc gcc-c++ gcc-plugin-devel.x86_64 gcc-plugin-devel.i686 libxml2-devel glibc glibc-devel glibc.i686 glibc-devel.i686 libgcc.i686 make suitesparse-static libtool-ltdl-devel glpk-devel gawk clang clang-devel llvm llvm-devel

ArchLinux
*************

The following packages have to be installed under ArchLinux to compile PandA:

- base-devel 
- multilib-devel 
- boost 
- iverilog 
- gcc-multilib 
- autoconf-archive 
- verilator 
- mpfi 
- clang 
- llvm 
- gcc 
- gcc-fortran 
- git 
- tar 
- suitesparse 
- glpk 
- readline

In a single line:
sudo pacman -S base-devel multilib-devel boost iverilog gcc-multilib autoconf-archive verilator mpfi clang llvm gcc gcc-fortran git tar suitesparse glpk readline

A working configure example is:
../configure --prefix=$HOME/panda --enable-verilator --enable-glpk --enable-opt --enable-flopoco --with-opt-level=fast

Mac OSX
*************
To compile PandA on MacOSX you need to install MacPorts (version 2.5.4 or later).

Here the list of commands required: 
     $sudo port -N -q install git autoconf autoconf-archive automake bison flex boost libtool SuiteSparse glpk iverilog clang-6.0 gcc8 mpfi coreutils
     $sudo cp /opt/local/bin/greadlink /opt/local/bin/readlink
     $git clone https://github.com/ferrandi/PandA-bambu.git
     $cd PandA-bambu
     $make -f Makefile.init
     $mkdir build
     $cd build && ../configure --prefix=$HOME/panda --enable-icarus --enable-glpk --enable-opt --enable-flopoco --with-opt-level=fast --with-clang6=/opt/local/bin/clang-mp-6.0 --with-gcc8=/opt/local/bin/gcc-mp-8 LDFLAGS=-L/opt/local/lib CPPFLAGS=-I/opt/local/include/ && make -j3 && make install && make clean



Notes on Ubuntu 13.04 64bit and XILINX ISE 14.6
*************

The vendor version of stdlibc++ does not have the correct versions of GLIBC for Ubuntu 13.04.
This is probably due to the fact that the tool is loading a mixture of system libraries and vendor libraries.
The solution is to replace the Xilinx version of stdlibc++ with the system version:
In case ISE is installed in /opt/Xilinx/14.6/ the solution is:
  $cd /opt/Xilinx/14.6/ISE_DS/ISE/lib/lin64/
  $sudo mv libstdc++.so.6 libstdc++.so.6.orig
  $sudo ln -s /usr/lib/x86_64-linux-gnu/libstdc++.so.6


Notes on Debian/Ubuntu and Lattice Diamond 3.9
*************

Lattice Diamond is currently distributed in rpm form for Linux distributions. To convert an rpm archive file into a .deb you can use the following instructions:
  $sudo apt-get install alien
  $sudo alien --scripts diamond_3_9-base_x64-99-2-x86_64-linux.rpm
once the package has been converted you can install with:
  $sudo dpkg -i diamond-3-9-base-x64_3.9-100_amd64.deb


Notes on NanoXplore tools
*************
The tested version of NanoXplore tools is the 2.9.4.
The configuration of the PandA framework may require to add this option to the configure script:
  --with-nanoxplore-bypass=<operating-system-distribution>
where possible value can be:
  x86_64_UBUNTU_14
  x86_64_UBUNTU_16
  x86_64_UBUNTU_18
  x86_64_DEBIAN_9
  x86_64_RHEL_6

Notes on Modelsim
*************

PandA/bambu supports both Modelsim full version and Student/Free version by adding --enable-modelsim option to the PandA configure.
When the Free/Student version of Modelsim is used, PandA/bambu should be configured by further adding the --disable-mentor-optimizer option. The Student/Free version does not have the optimization feature.


Notes on backend installation directories
*************

PandA/bambu automatically looks for the RTL synthesis/simulation backends once they are enabled. By default, it looks into specific directories of the filesystem.
For Altera it checks this set of directories /opt/altera/* /opt/intelFPGA/*
For Lattice it checks this set of directories /usr/local/diamond/*
For Mentor it checks this set of directories /opt/mentor and /opt/mentor/*
For Xilinx ISE it checks these sets of directories /opt/Xilinx/* and /opt/Xilinx/*/ISE*
For Xilinx VIVADO it checks this set of directories /opt/Xilinx/* and /opt/Xilinx/Vivado/*
For BRAVE FPGA NanoXplore it checks this set of directories /opt/NanoXplore/NXmap/*
Anyway, it is possible to specify a different position of the filesystem for the configuration script to look into.
This is the list of options that may be useful for this purpose:
  --with-altera-dir=DIR  where the root where ALTERA tools are installed
  --with-lattice-dir=DIR  where the root where LATTICE tools are installed
  --with-mentor-dir=DIR  where the root where MENTOR tools are installed
  --with-xilinx-root-dir=DIR  where the root where XILINX tools are installed
  --with-nanoxplore-dir=DIR  where the root where BRAVE FPGA NanoXplore tools are installed

\end{DoxyVerbInclude}
 