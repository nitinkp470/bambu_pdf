The data path describes a register transfer level design analogous to a schematic diagram. The data path is described in terms of modules and their interconnections.

{\bfseries Datapath}\+: The datapath ( $DP$) is a graph $DP(M_o\cup M_s \cup M_i,I)$ where
\begin{DoxyItemize}
\item a set $M = M_o\cup M_s\cup M_i$, whose elements, called modules, are the nodes of the graph, with
\begin{DoxyItemize}
\item a set  of {\itshape operational} modules like adders, multipliers and A\+L\+Us (see \hyperlink{src_HLS_module_binding_page}{Module binding}),
\item a set  of {\itshape storage} modules like registers and register files (see \hyperlink{src_HLS_registerAllocation_page}{Register allocation}),
\item a set $M_i$ of {\itshape interconnection} modules like multiplexers, demultiplexers busses and bus drivers (see src\+\_\+\+H\+L\+S\+\_\+datapath\+\_\+connection\+\_\+binding\+\_\+page);
\end{DoxyItemize}
\item an interconnection relation $I\subseteq M\times M$, whose elements are interconnection links. These are the edges of the datapath graph. 
\end{DoxyItemize}